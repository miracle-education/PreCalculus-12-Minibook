% Options for packages loaded elsewhere
\PassOptionsToPackage{unicode}{hyperref}
\PassOptionsToPackage{hyphens}{url}
%
\documentclass[
]{book}
\title{PreCalculus 12 Minibook}
\author{Joshua Zhang}
\date{}

\usepackage{amsmath,amssymb}
\usepackage{lmodern}
\usepackage{iftex}
\ifPDFTeX
  \usepackage[T1]{fontenc}
  \usepackage[utf8]{inputenc}
  \usepackage{textcomp} % provide euro and other symbols
\else % if luatex or xetex
  \usepackage{unicode-math}
  \defaultfontfeatures{Scale=MatchLowercase}
  \defaultfontfeatures[\rmfamily]{Ligatures=TeX,Scale=1}
\fi
% Use upquote if available, for straight quotes in verbatim environments
\IfFileExists{upquote.sty}{\usepackage{upquote}}{}
\IfFileExists{microtype.sty}{% use microtype if available
  \usepackage[]{microtype}
  \UseMicrotypeSet[protrusion]{basicmath} % disable protrusion for tt fonts
}{}
\makeatletter
\@ifundefined{KOMAClassName}{% if non-KOMA class
  \IfFileExists{parskip.sty}{%
    \usepackage{parskip}
  }{% else
    \setlength{\parindent}{0pt}
    \setlength{\parskip}{6pt plus 2pt minus 1pt}}
}{% if KOMA class
  \KOMAoptions{parskip=half}}
\makeatother
\usepackage{xcolor}
\IfFileExists{xurl.sty}{\usepackage{xurl}}{} % add URL line breaks if available
\IfFileExists{bookmark.sty}{\usepackage{bookmark}}{\usepackage{hyperref}}
\hypersetup{
  pdftitle={PreCalculus 12 Minibook},
  pdfauthor={Joshua Zhang},
  hidelinks,
  pdfcreator={LaTeX via pandoc}}
\urlstyle{same} % disable monospaced font for URLs
\usepackage{longtable,booktabs,array}
\usepackage{calc} % for calculating minipage widths
% Correct order of tables after \paragraph or \subparagraph
\usepackage{etoolbox}
\makeatletter
\patchcmd\longtable{\par}{\if@noskipsec\mbox{}\fi\par}{}{}
\makeatother
% Allow footnotes in longtable head/foot
\IfFileExists{footnotehyper.sty}{\usepackage{footnotehyper}}{\usepackage{footnote}}
\makesavenoteenv{longtable}
\usepackage{graphicx}
\makeatletter
\def\maxwidth{\ifdim\Gin@nat@width>\linewidth\linewidth\else\Gin@nat@width\fi}
\def\maxheight{\ifdim\Gin@nat@height>\textheight\textheight\else\Gin@nat@height\fi}
\makeatother
% Scale images if necessary, so that they will not overflow the page
% margins by default, and it is still possible to overwrite the defaults
% using explicit options in \includegraphics[width, height, ...]{}
\setkeys{Gin}{width=\maxwidth,height=\maxheight,keepaspectratio}
% Set default figure placement to htbp
\makeatletter
\def\fps@figure{htbp}
\makeatother
\setlength{\emergencystretch}{3em} % prevent overfull lines
\providecommand{\tightlist}{%
  \setlength{\itemsep}{0pt}\setlength{\parskip}{0pt}}
\setcounter{secnumdepth}{5}
\usepackage{booktabs}
\ifLuaTeX
  \usepackage{selnolig}  % disable illegal ligatures
\fi

\usepackage{amsthm}
\newtheorem{theorem}{Theorem}[chapter]
\newtheorem{lemma}{Lemma}[chapter]
\newtheorem{corollary}{Corollary}[chapter]
\newtheorem{proposition}{Proposition}[chapter]
\newtheorem{conjecture}{Conjecture}[chapter]
\theoremstyle{definition}
\newtheorem{definition}{Definition}[chapter]
\theoremstyle{definition}
\newtheorem{example}{Example}[chapter]
\theoremstyle{definition}
\newtheorem{exercise}{Exercise}[chapter]
\theoremstyle{definition}
\newtheorem{hypothesis}{Hypothesis}[chapter]
\theoremstyle{remark}
\newtheorem*{remark}{Remark}
\newtheorem*{solution}{Solution}
\begin{document}
\maketitle

{
\setcounter{tocdepth}{1}
\tableofcontents
}
\hypertarget{preface}{%
\chapter*{Preface}\label{preface}}
\addcontentsline{toc}{chapter}{Preface}

This minibook summarizes the key concepts and common examples in Pre-Calculus 12 to aid students in learning and reviewing. The content follows the curriculum in BC, Canada.

\hypertarget{motivation}{%
\section{Motivation}\label{motivation}}

This minibook \ldots{}

\hypertarget{suggestions-to-the-readers}{%
\section{Suggestions to the Readers}\label{suggestions-to-the-readers}}

\ldots{}

\hypertarget{about-the-author}{%
\section{About the Author}\label{about-the-author}}

\ldots{}

\hypertarget{how-to-learn-math}{%
\section{How to Learn Math?}\label{how-to-learn-math}}

\ldots{}

\hypertarget{reviewprerequisite}{%
\chapter*{Review/Prerequisite}\label{reviewprerequisite}}
\addcontentsline{toc}{chapter}{Review/Prerequisite}

This chapter summarizes what you are expected to know before reading this book, or take pre-calculus 12.

\hypertarget{order-of-operations}{%
\section{Order of Operations}\label{order-of-operations}}

\ldots{}

\hypertarget{set}{%
\section{Set}\label{set}}

\ldots{}

\hypertarget{function-transformations}{%
\chapter{Function Transformations}\label{function-transformations}}

This chapter introduces \ldots{}

\hypertarget{polynomials-and-polynomial-functions}{%
\chapter{Polynomials and Polynomial Functions}\label{polynomials-and-polynomial-functions}}

\ldots{}

\hypertarget{rational-radical-and-absolute-value-functions}{%
\chapter{Rational, Radical, and Absolute Value Functions}\label{rational-radical-and-absolute-value-functions}}

\hypertarget{proportionality}{%
\section{Proportionality}\label{proportionality}}

\hypertarget{direct-proportionality}{%
\subsection{Direct Proportionality}\label{direct-proportionality}}

\ldots{}

\hypertarget{inverse-proportionality}{%
\subsection{Inverse Proportionality}\label{inverse-proportionality}}

\ldots{}

\hypertarget{exponenetial-and-logarithmic-fucnitons}{%
\chapter{Exponenetial and Logarithmic Fucnitons}\label{exponenetial-and-logarithmic-fucnitons}}

\hypertarget{equations}{%
\section{Equations}\label{equations}}

Here is an equation.

\begin{equation} 
  f\left(k\right) = \binom{n}{k} p^k\left(1-p\right)^{n-k}
  \label{eq:binom}
\end{equation}

You may refer to using \texttt{\textbackslash{}@ref(eq:binom)}, like see Equation \eqref{eq:binom}.

\hypertarget{theorems-and-proofs}{%
\section{Theorems and proofs}\label{theorems-and-proofs}}

Labeled theorems can be referenced in text using \texttt{\textbackslash{}@ref(thm:tri)}, for example, check out this smart theorem \ref{thm:tri}.

\begin{theorem}
\protect\hypertarget{thm:tri}{}\label{thm:tri}For a right triangle, if \(c\) denotes the \emph{length} of the hypotenuse
and \(a\) and \(b\) denote the lengths of the \textbf{other} two sides, we have
\[a^2 + b^2 = c^2\]
\end{theorem}

Read more here \url{https://bookdown.org/yihui/bookdown/markdown-extensions-by-bookdown.html}.

\hypertarget{callout-blocks}{%
\section{Callout blocks}\label{callout-blocks}}

The R Markdown Cookbook provides more help on how to use custom blocks to design your own callouts: \url{https://bookdown.org/yihui/rmarkdown-cookbook/custom-blocks.html}

\hypertarget{geometric-sequences-and-series}{%
\chapter{Geometric Sequences and Series}\label{geometric-sequences-and-series}}

\ldots{}

\hypertarget{trigonometric-functions}{%
\chapter{Trigonometric Functions}\label{trigonometric-functions}}

\ldots{}

\hypertarget{trigonometric-identities}{%
\chapter{Trigonometric Identities}\label{trigonometric-identities}}

\ldots{}

\hypertarget{conics}{%
\chapter{Conics}\label{conics}}

\ldots{}

\hypertarget{other-topics}{%
\chapter{Other Topics}\label{other-topics}}

\ldots{}

\end{document}
